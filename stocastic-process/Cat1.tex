% Options for packages loaded elsewhere
\PassOptionsToPackage{unicode}{hyperref}
\PassOptionsToPackage{hyphens}{url}
%
\documentclass[
]{article}
\title{CAT 1}
\author{79546 Stephen Ng'etich}
\date{}

\usepackage{amsmath,amssymb}
\usepackage{lmodern}
\usepackage{iftex}
\ifPDFTeX
  \usepackage[T1]{fontenc}
  \usepackage[utf8]{inputenc}
  \usepackage{textcomp} % provide euro and other symbols
\else % if luatex or xetex
  \usepackage{unicode-math}
  \defaultfontfeatures{Scale=MatchLowercase}
  \defaultfontfeatures[\rmfamily]{Ligatures=TeX,Scale=1}
\fi
% Use upquote if available, for straight quotes in verbatim environments
\IfFileExists{upquote.sty}{\usepackage{upquote}}{}
\IfFileExists{microtype.sty}{% use microtype if available
  \usepackage[]{microtype}
  \UseMicrotypeSet[protrusion]{basicmath} % disable protrusion for tt fonts
}{}
\makeatletter
\@ifundefined{KOMAClassName}{% if non-KOMA class
  \IfFileExists{parskip.sty}{%
    \usepackage{parskip}
  }{% else
    \setlength{\parindent}{0pt}
    \setlength{\parskip}{6pt plus 2pt minus 1pt}}
}{% if KOMA class
  \KOMAoptions{parskip=half}}
\makeatother
\usepackage{xcolor}
\IfFileExists{xurl.sty}{\usepackage{xurl}}{} % add URL line breaks if available
\IfFileExists{bookmark.sty}{\usepackage{bookmark}}{\usepackage{hyperref}}
\hypersetup{
  pdftitle={CAT 1},
  pdfauthor={79546 Stephen Ng'etich},
  hidelinks,
  pdfcreator={LaTeX via pandoc}}
\urlstyle{same} % disable monospaced font for URLs
\usepackage[margin=1in]{geometry}
\usepackage{color}
\usepackage{fancyvrb}
\newcommand{\VerbBar}{|}
\newcommand{\VERB}{\Verb[commandchars=\\\{\}]}
\DefineVerbatimEnvironment{Highlighting}{Verbatim}{commandchars=\\\{\}}
% Add ',fontsize=\small' for more characters per line
\usepackage{framed}
\definecolor{shadecolor}{RGB}{248,248,248}
\newenvironment{Shaded}{\begin{snugshade}}{\end{snugshade}}
\newcommand{\AlertTok}[1]{\textcolor[rgb]{0.94,0.16,0.16}{#1}}
\newcommand{\AnnotationTok}[1]{\textcolor[rgb]{0.56,0.35,0.01}{\textbf{\textit{#1}}}}
\newcommand{\AttributeTok}[1]{\textcolor[rgb]{0.77,0.63,0.00}{#1}}
\newcommand{\BaseNTok}[1]{\textcolor[rgb]{0.00,0.00,0.81}{#1}}
\newcommand{\BuiltInTok}[1]{#1}
\newcommand{\CharTok}[1]{\textcolor[rgb]{0.31,0.60,0.02}{#1}}
\newcommand{\CommentTok}[1]{\textcolor[rgb]{0.56,0.35,0.01}{\textit{#1}}}
\newcommand{\CommentVarTok}[1]{\textcolor[rgb]{0.56,0.35,0.01}{\textbf{\textit{#1}}}}
\newcommand{\ConstantTok}[1]{\textcolor[rgb]{0.00,0.00,0.00}{#1}}
\newcommand{\ControlFlowTok}[1]{\textcolor[rgb]{0.13,0.29,0.53}{\textbf{#1}}}
\newcommand{\DataTypeTok}[1]{\textcolor[rgb]{0.13,0.29,0.53}{#1}}
\newcommand{\DecValTok}[1]{\textcolor[rgb]{0.00,0.00,0.81}{#1}}
\newcommand{\DocumentationTok}[1]{\textcolor[rgb]{0.56,0.35,0.01}{\textbf{\textit{#1}}}}
\newcommand{\ErrorTok}[1]{\textcolor[rgb]{0.64,0.00,0.00}{\textbf{#1}}}
\newcommand{\ExtensionTok}[1]{#1}
\newcommand{\FloatTok}[1]{\textcolor[rgb]{0.00,0.00,0.81}{#1}}
\newcommand{\FunctionTok}[1]{\textcolor[rgb]{0.00,0.00,0.00}{#1}}
\newcommand{\ImportTok}[1]{#1}
\newcommand{\InformationTok}[1]{\textcolor[rgb]{0.56,0.35,0.01}{\textbf{\textit{#1}}}}
\newcommand{\KeywordTok}[1]{\textcolor[rgb]{0.13,0.29,0.53}{\textbf{#1}}}
\newcommand{\NormalTok}[1]{#1}
\newcommand{\OperatorTok}[1]{\textcolor[rgb]{0.81,0.36,0.00}{\textbf{#1}}}
\newcommand{\OtherTok}[1]{\textcolor[rgb]{0.56,0.35,0.01}{#1}}
\newcommand{\PreprocessorTok}[1]{\textcolor[rgb]{0.56,0.35,0.01}{\textit{#1}}}
\newcommand{\RegionMarkerTok}[1]{#1}
\newcommand{\SpecialCharTok}[1]{\textcolor[rgb]{0.00,0.00,0.00}{#1}}
\newcommand{\SpecialStringTok}[1]{\textcolor[rgb]{0.31,0.60,0.02}{#1}}
\newcommand{\StringTok}[1]{\textcolor[rgb]{0.31,0.60,0.02}{#1}}
\newcommand{\VariableTok}[1]{\textcolor[rgb]{0.00,0.00,0.00}{#1}}
\newcommand{\VerbatimStringTok}[1]{\textcolor[rgb]{0.31,0.60,0.02}{#1}}
\newcommand{\WarningTok}[1]{\textcolor[rgb]{0.56,0.35,0.01}{\textbf{\textit{#1}}}}
\usepackage{graphicx}
\makeatletter
\def\maxwidth{\ifdim\Gin@nat@width>\linewidth\linewidth\else\Gin@nat@width\fi}
\def\maxheight{\ifdim\Gin@nat@height>\textheight\textheight\else\Gin@nat@height\fi}
\makeatother
% Scale images if necessary, so that they will not overflow the page
% margins by default, and it is still possible to overwrite the defaults
% using explicit options in \includegraphics[width, height, ...]{}
\setkeys{Gin}{width=\maxwidth,height=\maxheight,keepaspectratio}
% Set default figure placement to htbp
\makeatletter
\def\fps@figure{htbp}
\makeatother
\setlength{\emergencystretch}{3em} % prevent overfull lines
\providecommand{\tightlist}{%
  \setlength{\itemsep}{0pt}\setlength{\parskip}{0pt}}
\setcounter{secnumdepth}{-\maxdimen} % remove section numbering
\usepackage{amsmath}
\usepackage{xspace}
\ifLuaTeX
  \usepackage{selnolig}  % disable illegal ligatures
\fi

\begin{document}
\maketitle

\hypertarget{q1}{%
\section{Q1}\label{q1}}

\hypertarget{a-peter-bets-5-dollars-each-time-timid-strategy.}{%
\subsection{a) Peter bets 5 dollars each time (timid
strategy).}\label{a-peter-bets-5-dollars-each-time-timid-strategy.}}

\[p=0.55,q=0.45,N=8,z=5\]

\begin{Shaded}
\begin{Highlighting}[]
\NormalTok{p }\OtherTok{=} \FloatTok{0.55}
\NormalTok{q }\OtherTok{=} \FloatTok{0.45}
\NormalTok{N }\OtherTok{=} \DecValTok{8}
\NormalTok{i }\OtherTok{=} \DecValTok{5}

\NormalTok{timid\_results }\OtherTok{=}\NormalTok{ (}\DecValTok{1}\SpecialCharTok{{-}}\NormalTok{(q}\SpecialCharTok{/}\NormalTok{p)}\SpecialCharTok{\^{}}\NormalTok{i) }\SpecialCharTok{/}\NormalTok{ (}\DecValTok{1}\SpecialCharTok{{-}}\NormalTok{(q}\SpecialCharTok{/}\NormalTok{p)}\SpecialCharTok{\^{}}\NormalTok{N)}

\NormalTok{timid\_results}
\end{Highlighting}
\end{Shaded}

\begin{verbatim}
## [1] 0.7924987
\end{verbatim}

\hypertarget{b-peter-uses-a-bold-strategy}{%
\subsection{b) Peter uses a bold
strategy}\label{b-peter-uses-a-bold-strategy}}

Since peter uses a bold strategy i.e play with the money earned in each
round.in this case he will play 4 round to get to 8 dollars.

\begin{Shaded}
\begin{Highlighting}[]
\NormalTok{q\_0 }\OtherTok{\textless{}{-}} \DecValTok{0}
\NormalTok{q\_8 }\OtherTok{\textless{}{-}} \DecValTok{1}

\NormalTok{p }\OtherTok{\textless{}{-}} \FloatTok{0.55}
\NormalTok{q }\OtherTok{\textless{}{-}} \FloatTok{0.45}

\NormalTok{q\_4 }\OtherTok{\textless{}{-}}\NormalTok{ p}\SpecialCharTok{*}\NormalTok{(q\_8)}\SpecialCharTok{+}\NormalTok{q}\SpecialCharTok{*}\NormalTok{(q\_0)}
\NormalTok{q\_2 }\OtherTok{\textless{}{-}}\NormalTok{ p}\SpecialCharTok{*}\NormalTok{(q\_4)}\SpecialCharTok{+}\NormalTok{q}\SpecialCharTok{*}\NormalTok{(q\_0)}
\NormalTok{result\_bold }\OtherTok{\textless{}{-}}\NormalTok{ p}\SpecialCharTok{*}\NormalTok{(q\_2)}\SpecialCharTok{+}\NormalTok{q}\SpecialCharTok{*}\NormalTok{(q\_0)}

\NormalTok{result\_bold}
\end{Highlighting}
\end{Shaded}

\begin{verbatim}
## [1] 0.166375
\end{verbatim}

\hypertarget{c-which-strategy-is-better-timid-or-bold}{%
\subsection{c) Which strategy is better timid or
bold}\label{c-which-strategy-is-better-timid-or-bold}}

Timid Strategy is better than the bold strategy

\hypertarget{q2}{%
\section{Q2}\label{q2}}

The gambler's ruin problem: player A starts with fortune \(j\) dollars
and bets 1 dollar until he either loses all the fortune or reach a \(N\)
dollars fortune and then quit.

let \(A_j\) represent the the event that player A win starting with j
dollars.
\[ x_j = P(A_j) = \text{probability to win starting with a j} = \text{probability to reach N before reaching 0 starting from j}\]

Using conditional probability and condition on what happens at the first
game, win,lose or tie,every game will have
\[P(win)=p, P(lose)=q,P(tie)=r\] which yields
\[\begin{aligned} x_j & = P(A_j) \\ & = P(A_j|win)P(win) + P(A_j|lose)P(lose)+P(A_j|tie)P(tie) \\ & =  x_j \times p + x_j \times q + x_j \times r \end{aligned}\]
note that \(x_0 = P(A_0) = 0\) since their nothing more to gamble and
\(x_N = P(A_N)\) since player A has reached their goal and then stop
playing hence the equation \(p+q+r=1\)\\
Gamblers Ruin \[px_{j+1} - (p+q)x_j+qx_{j-1} = 0, x(0)=0,x(N)=1\]

.Rewriting the equation in second order equation with \(x_j = \alpha^j\)
we find the quadratic equation \[p\alpha^2 - (p+q)\alpha+q=0\] with
solutions
\[\alpha = \frac{p+q \pm \sqrt{(p+q)^2-4pq}}{2p} = \frac{p+q \pm \sqrt{p^2+q^2-4pq}}{2p} = \frac{p+q \pm\sqrt{(p+q)^2}}{2p} = \bigg\{ \begin{array}{ll}1 \\ q/p\end{array} \]
if \(p \ne q\) there arer two solutions and so the general is given by
\[x_n = C_1 1^n + C_2 \bigg(\frac{q}{p} \bigg)^n\]. Using
\(X_o=0 \text{ (lose) },X_N=1 \text{ (win) }\) to determine the
constants \(C_1\) and \(C_2\).
\[0=C_1+C_2,  and 1=C_1+C_2\bigg(\frac{q}{p}\bigg)^N\] which gives
\[C_1 = -C_2 = \bigg(1-\bigg(\frac{q}{p}\bigg)\bigg)^{-1}\] hence
\[ \text{Gambler's ruin probabilities: } x_n = \frac{1-(q/p)^n}{1-(p/q)^N} \, \, p\ne q\]

if the game is fair i.e.~\(p=q\) the gambler ruin probabilities \(X_j\)
simplify to \[x(j+1)-2x(j)+x(j-1)\] which gieve the quadratic equation
\[\alpha^2 - 2\alpha +1\] with the only root \(\alpha =1\).The genral
solution is \[x(j)=C_1+C_2j\] with \$x(0) = 0 \$ and \(x(N) =1\) which
results
\[ \text{Gambler's ruin probabilities: } x_n = \frac{j}{N} \, \, if \,\,p = q\]

in summary the game is fair \[
\begin{aligned} 
\text{fair  if } p=q \\ 
\text{subfair if  } p < q \\ 
\text{superfair if } p > q 
\end{aligned}
\]

\hypertarget{q3}{%
\section{Q3}\label{q3}}

Tom starts with \$5, and p = 0.63: What is the probability that Tom
obtains a fortune of N = 12 without going broke?

\[i=5,N=12 \text{ and } q=1-p=0.37\] hence \[\frac{q}{p}=\frac{37}{63}\]
\[P_2 = \frac{1-(37/63)^5}{1-(37/63)^{12}} = \frac{0.93012}{0.99832} = 0.9317\]
What is the probability that Tom will become infinitely rich?

\[1-(q/p)^i = 1-(37/63)^5 = 0.93012 \]

If Tom instead started with i = \$2, what is the probability that he
would go broke?

\[ \text{The probability he becomes rich is } 1-(q/p)^i  = 1-(37/63)^2= 0.6550\]

\hypertarget{q4}{%
\section{Q4}\label{q4}}

The probability that the stock goes up by 15 before going down by
6.computing \(p(a)\) \[p=0.6,q=1-p=0.4,a=15,b=6\]
\[p(a)=\frac{1-(\frac{q}{p})^b}{1-(\frac{q}{p})^{a+b}} = \frac{1-(\frac{0.4}{0.6})^6}{1-(\frac{0.4}{0.6})^{15+6}} = \frac{0.9122}{0.9998} =0.9914 \]

\end{document}
